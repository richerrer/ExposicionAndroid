\documentclass[12pt,letterpaper]{article}

\usepackage[right=2cm,left=3cm,top=2cm,bottom=2cm,headsep=0cm,footskip=0.5cm]{geometry}
\usepackage{graphicx}
\usepackage[spanish]{babel} % Para separar correctamente las palabras
\usepackage[utf8]{inputenc} % Este paquete permite poner acentos y e?es usando codificaci?n utf-8

\begin{document}


\section{Aplicación: MaxSec App}
\subsection{Autores}

* Mira Rodríguez Raúl Alberto

* Romero Triviño Jose Andrés

* Maya Herrera Ricardo David


\section{Manual de Usuario}
\subsection{Comenzando a usar MaxSec}

Antes de descargar e instalar MaxSec, por favor asegurarse de que la aplicacion soporte el tipo de telefono.
Una vez instalado, para usar por primera vez la aplicacion, el movil debera de estar conectado a internet (para el uso del gps ypara poder el enviar el email correspondiente de presentarse el caso).

\subsection{Configuracion Inicial}

 \begin{itemize}

\item{Abra la aplicacion y seleccione el boton Options para hacer una configuracion inicial}

\item{Ya en Options llene el campo Mail con su correo electronico}

\item{El campo Old Password se lo deja vacio ya que como es la primera vez que se utiliza la aplicacion no tenemos ninguna contraseña anterior}

\item{En New Password ingrese la contraseña de desbloqueo personalizada}

\item{Finalmente se da tap en el boton Aceptar para confirmar y guardar la configuracion}

\item{De lo contrario dar tap en el boton Cancelar para regresar a la pantalla principal}

\end{itemize}

\subsection{Bloqueo del Telefono}
Luego de haber efectuado la configuracion inicial exitosamente podemos encender la aplicacion de tal manera que se active la alarma. Para esto deberemos seguir los siguientes pasos:

 \begin{itemize}

\item{En la pantalla principal seleccionar Run}
\item{Ya en Run podremos ver el correo electronico que ingresamos anteriormente asi como tambien el sonido de alarma que esta seleccionado}
\item{Para activar la alarma simplemente se da tap en Turn On}
\item{Luego de haber hecho esto la aplicacion seguira ejecutandose en segundo plano y el usuario tendra un tiempo de 15 segundos para dejar el movil en un lugar donde no se mueva.}

\end{itemize}



\subsection{Desbloqueo del Telefono}
Para desbloquear el telefono se realizan los siguientes pasos:

\subsection{Cambiar contraseña o Correo}
En caso de que el usuario desee cambiar su contraseña actual por una nueva debera seguir los siguientes pasos:


\subsection{Cambiar Sonido de la Alarma}


\end{document}