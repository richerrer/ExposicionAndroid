\documentclass[12pt,letterpaper]{article}

\usepackage[right=2cm,left=3cm,top=2cm,bottom=2cm,headsep=0cm,footskip=0.5cm]{geometry}
\usepackage{graphicx}
\usepackage[spanish]{babel} % Para separar correctamente las palabras
\usepackage[utf8]{inputenc} % Este paquete permite poner acentos y e?es usando codificaci?n utf-8

\begin{document}



\section{Aplicación: MaxSec App}
\subsection{Autores}

* Mira Rodríguez Raúl Alberto

* Romero Triviño Jose Andrés

* Maya Herrera Ricardo David

\section{Manual de Usuario}
\subsection{Comenzando a usar MaxSec}

Antes de descargar e instalar MaxSec, por favor asegurarse de que la aplicacion soporte el tipo de telefono.
Una vez instalado, para usar por primera vez la aplicacion, el movil debera de estar conectado a internet (para el uso del gps ypara poder el enviar el email correspondiente de presentarse el caso).

\subsection{Configuracion Inicial}

 \begin{itemize}

\item{Abra la aplicacion y seleccione el boton Options para hacer una configuracion inicial}

\item{Ya en Options llene el campo Mail con su correo electronico}

\item{El campo Old Password se lo deja vacio ya que como es la primera vez que se utiliza la aplicacion no tenemos ninguna contraseña anterior}

\item{En New Password ingrese la contraseña de desbloqueo que usted desee}

\item{Finalmente se da tap en el boton Aceptar para confirmar y guardar la configuracion}

\item{De lo contrario dar tap en el boton Cancelar para regresar a la pantalla principal}

\end{itemize}

\subsection{Bloqueo del Telefono}
Luego de haber efectuado la configuracion inicial exitosamente podemos encender la aplicacion de tal manera que se active la alarma. Para esto deberemos seguir los siguientes pasos:

 \begin{itemize}

\item{En la pantalla principal seleccionar la opcion Run}
\item{Ya en Run podremos ver el correo electronico que ingresamos anteriormente asi como tambien el sonido de alarma que esta seleccionado}
\item{Para activar la alarma simplemente se da tap en boton Turn On}
\item{Luego de haber hecho esto la aplicacion seguira ejecutandose en segundo plano y el usuario tendra un tiempo de 15 segundos para dejar el movil en un lugar donde no se mueva.}

\end{itemize}



\subsection{Desbloqueo del Telefono}
Ya bloqueado el telefono, si alguien lo mueve de su posicion automaticamente se mostrara una pantalla donde se solicitara que se ingrese la contraseña de desbloqueo. Luego de ingresar la contraseña y dar tap en el boton Aceptar se podrian presentar los siguientes escenarios:

 \begin{itemize}

\item{Si la contraseña es correcta el telefono se habra desbloqueado exitosamente}
\item{Si la contraseña es incorrecta se empezara a emitir el sonido de alarma}
\item{Si se da tap en el boton Cancelar tambien se emitira el sonido de alarma}

\end{itemize}



\subsection{Cambiar contraseña o Correo}
En caso de que el usuario desee cambiar su contraseña actual por una nueva debera seguir los siguientes pasos:

 \begin{itemize}

\item{Abra la aplicacion y seleccione el boton Options.}
\item{Ya en Options llene el campo Mail reemplazamos el correo electronico que esta escrito por el nuevo que el usuario desee, de ser este el caso, de lo contrario este campo se lo deja igual }
\item{El campo Old Password se escribe la contraseña que estamos usando actualmente y queremos cambiar}
\item{En el item New Password ingrese la nueva contraseña de desbloqueo que usted desee}
\item{Finalmente se da tap en el boton Aceptar para confirmar y guardar la configuracion}
\item{De lo contrario dar tap en el boton Cancelar para regresar a la pantalla principal}


\end{itemize}


\subsection{Cambiar Sonido de la Alarma}
El sonido de alarma se lo podra cambiar al momento en que se va a activar la aplicacion Bloqueo del telefono, en esta pantalla tenemos el campo Sound que nos permitira escoger el sonido de alarma de nuestra preferencia


\end{document}