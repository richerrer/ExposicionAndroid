\documentclass{beamer}
\usepackage[spanish]{babel} % Para separar correctamente las palabras
\usepackage[utf8]{inputenc} % Este paquete permite poner acentos y e?es usando codificaci?n utf-8

\usetheme{Berlin}
\title{MaxSec APP}
\author{Raul Mira \\ Jose Romero \\ Ricardo Maya}



%Comienzo del documento

\begin{document}
  \begin{frame}
            \includegraphics[height=0.2\textheight]{C:/Users/Ricardo/Documents/beamer/espol} \hspace*{7.3cm}
             \includegraphics[height=0.22\textheight]{C:/Users/Ricardo/Documents/beamer/android}
            \titlepage
            \scriptsize
                 \begin{center}
                     Sitemas Multimedia \\
                      Escuela Superior Politecnica del litoral \\
                  \end{center}
     \end{frame}

  \begin{frame}
    
   
    
    \scriptsize
    
    \begin{center}
       \textcolor{blue}{APLICACION PARA ANDROID \\}
      MaxSec App \\
    \end{center}
    
    
    MaxSec APP ,es una aplicacion antirrobo para tu smartphone que incluye una alarma. Cuando alguien intente robar tu celular, la aplicación al detectar el movimiento del movil,
pedira una contraseña que tu habras definido y que en caso de no ser ingresada correctamente en un corto lapso de tiempo, comenzara a sonar una ruidosa alarma y
enviara la posicion geografica adjunto una fotografia de el lugar o persona donde ocurre el incidente (robo), y tambien un correo electronico a un familiar indicando lo sucedido
    
  \end{frame}
  
   \begin{frame}
      
       \begin{center}
        \textcolor{blue}{OBJETIVOS\\}
       \end{center}

     \begin{enumerate}
      \item<1-> Evitar que alguien coja tu telefono sin tu permiso (familiar ,amigo bromista)
      \item<1-> Evitar el robo de tu dispositivo mediante una alarma sensible al movimiento
      \item<1-> Llamar la atencion a la gente de alrededor en caso que suceda
      \end{enumerate}
     \end{frame}   

      \scriptsize
    \begin{frame}
      \begin{center}
       \textcolor{blue}{ CARACTERISTICAS GENERALES\\}
      \end{center}
      
      \begin{enumerate}
      \item<1-> Plataforma .- Android
      \item<1-> Lenguaje .- Java.
      \item<1-> Version No. 1
      \item<1-> Uso de camara dispositivo Android
      \item<1-> GPS Global Positioning System
      \item<1-> Activacion mediante el Acelerometro
      \item<1-> Tiempo de desarrollo .- aprox. 5 semanas
      \item<1-> Programas a utilizar .- Eclipse IDE for Java developers
      \item<1-> Android SDK (android 2.1)
      \item<1-> Idioma{\tiny } Espanol
     \end{enumerate}
      
    \end{frame}
  
  \begin{frame}
       \scriptsize
        \begin{center}
                 \textcolor{red}{IMPLEMENTACION DE LA APLICACION\\}
         \end{center}
          \textcolor{blue}{1.}
          Podemos observar la pantalla principal de nuestra aplicación con los botones indicados en la imagen. En caso de no estar encendido el gps se mostrará un mensaje para que el usuario lo encienda. Despues de encendido tenemos que esperar que el gps localice nuestra posición antes de presionar el boton run.
           \begin{center}
                  \includegraphics[height=0.3\textheight]{C:/Users/Ricardo/Documents/GitHub/exposicionFinal/max1} 
         \end{center}

        \textcolor{blue}{2.}
         Al presionar el boton Run, este lanza otra ventana con el mail por defecto al que la  aplicación mandará el correo, y su password  para que no se ejecute ninguna alarma.
           \begin{center}
                 \includegraphics[height=0.25\textheight]{C:/Users/Ricardo/Documents/GitHub/exposicionFinal/max2} 
         \end{center}
  \end{frame}

\begin{frame}
       \scriptsize
        
          \textcolor{blue}{3.}
       Al presionar el boton de encendido, la aplicación se minimiza esperando un lapso de tiempo para que el usuario pueda dejar inmovil el dispositivo.
           \begin{center}
                 \includegraphics[height=0.3\textheight]{C:/Users/Ricardo/Documents/GitHub/exposicionFinal/max4} 
         \end{center}

        \textcolor{blue}{4.}
         El dispositivo al sentir algún movimiento lanza una ventana de desbloqueo la cual se encargará de tomar una foto y usar el gps para enviar al correo deseado (todo esto en segundo plano), siempre y cuando el código sea incorrecto la aplicación tomará fotos y seguirá bloqueada hasta que se digite correctamente la contraseña.
           \begin{center}
                   \includegraphics[height=0.3\textheight]{C:/Users/Ricardo/Documents/GitHub/exposicionFinal/max3} 
         \end{center}
  \end{frame}
    
\begin{frame}
       \scriptsize
   \textcolor{blue}{5.}
        .Aqui podemos observar el correo que se envió al mail configurado, en el cual se muestra el archivo de la imagen, la longitud, la latitud y la calle en la que fue tomada la foto.
            \begin{center}
                 \includegraphics[height=0.3\textheight]{C:/Users/Ricardo/Documents/GitHub/exposicionFinal/max6} \ \ \ \
 \includegraphics[height=0.3\textheight]{C:/Users/Ricardo/Documents/GitHub/exposicionFinal/max7} 
            \end{center}
      
 \textcolor{blue}{6.}
        En esta pantalla se puede configurar a que correo desea el usuario que se envie la imagen, también puede cambiar el código siempre y cuando sepa cual fue el anterior.
           \begin{center}
                 \includegraphics[height=0.3\textheight]{C:/Users/Ricardo/Documents/GitHub/exposicionFinal/max5} 
         \end{center}
  \end{frame}
    
  
\end{document}

